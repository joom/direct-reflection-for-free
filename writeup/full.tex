\documentclass[format=acmsmall, review=false, screen=true]{acmart}
\settopmatter{printacmref=false} % Removes citation information below abstract
\renewcommand\footnotetextcopyrightpermission[1]{} % removes footnote with conference information in first column
\pagestyle{plain} % removes running headers

% Copyright
\setcopyright{none}
%\setcopyright{acmcopyright}
%\setcopyright{acmlicensed}
%\setcopyright{rightsretained}
%\setcopyright{usgov}
%\setcopyright{usgovmixed}
%\setcopyright{cagov}
%\setcopyright{cagovmixed}

% DOI
\acmDOI{}

% ISBN
\acmISBN{}

%Conference
\acmConference[ICFP '20]{International Conference on Functional Programming}{August 2020}{Jersey City, New Jersey, USA}
\acmBooktitle{International Conference on Functional Programming}
\acmYear{2020}
\copyrightyear{2020}
\acmPrice{15.00}
\acmSubmissionID{}

\usepackage{enumerate, enumitem}
\usepackage{fancyvrb}
\fvset{commandchars=\\\{\}}
\usepackage{xcolor}
\usepackage{setspace}
\usepackage{textgreek}
\usepackage{todonotes}
\usepackage{tikz-cd}
\usepackage{adjustbox}
\usepackage{multicol}
% Code
% in REPL run `:pp latex 80 <expr>` to get the colored verbatim

\usepackage{amssymb}
\newcommand\fakeslant[1]{%
  \special{pdf: literal 1 0 0.167 1 0 0 cm}#1\special{pdf: literal 1 0 -0.167 1 0 0 cm}}
\definecolor{CodeRed}{HTML}{BC0045}
\definecolor{CodeBlue}{HTML}{0000d6}
\definecolor{CodeGreen}{HTML}{008b00}
\definecolor{CodeLilac}{HTML}{AC22BF}
\definecolor{CodeYellow}{HTML}{E16800}
\definecolor{CodeLightBlue}{HTML}{A4CDFF}
\newcommand{\CodeData}[1]{\textcolor{CodeRed}{#1}}
\newcommand{\CodeType}[1]{\textcolor{CodeBlue}{#1}}
\newcommand{\CodeBound}[1]{\textcolor{CodeLilac}{\fakeslant{#1}}}
\newcommand{\CodeFunction}[1]{\textcolor{CodeGreen}{#1}}
\newcommand{\CodeMetavar}[1]{\textcolor{CodeYellow}{#1}}
\newcommand{\CodeKeyword}[1]{{\textbf{#1}}}
\newcommand{\CodeImplicit}[1]{{\itshape \CodeBound{#1}}}
\newcommand{\ty}[1]{\CodeType{\texttt{#1}}}
\newcommand{\kw}[1]{\CodeKeyword{\texttt{#1}}}
\newcommand{\fn}[1]{\CodeFunction{\texttt{#1}}}
\newcommand{\dt}[1]{\CodeData{\texttt{#1}}}
\newcommand{\bn}[1]{\CodeBound{\texttt{#1}}}
\newcommand{\cm}[1]{\textcolor{darkgray}{\texttt{#1}}}
\newcommand{\hole}[1]{\CodeMetavar{\texttt{?}\CodeMetavar{\texttt{#1}}}}
\newcommand{\Elab}{\ty{Elab}}
\newcommand{\IO}{\ty{IO}}
\newcommand{\String}{\ty{String}}
\newcommand{\TT}{\ty{TT}}
\newcommand{\Raw}{\ty{Raw}}
\newcommand{\Editorable}{\ty{Editorable}}
\newcommand{\TyDecl}{\ty{TyDecl}}
\newcommand{\FunDefn}{\ty{FunDefn}}
\newcommand{\FunClause}{\ty{FunClause}}
\newcommand{\Edit}{\ty{Edit}}
\newcommand{\sexp}{\mbox{S-expression}} %don't hyphenate
\newcommand{\mt}[1]{\mbox{\texttt{#1}}}
\newcommand{\Prim}[1]{\dt{Prim\_\_#1}}
\newcommand{\qt}[1]{\kw{\`{}(}{#1}\kw{)}} %quotation `(...)
\newcommand{\antiqt}{\kw{\textasciitilde}} %antiquotation ~

\newcommand{\lam}{\texorpdfstring{\textlambda}{lambda}}
\newcommand{\Lam}{\texorpdfstring{\textlambda}{Lambda}}
\newcommand{\moo}{\texorpdfstring{\textmu}{mu}}
% \newcommand{\lam}{\textlambda}
% \newcommand{\Lam}{\textlambda}
% \newcommand{\moo}{\textmu}

% Replicating selecting regions in the editor:
% \newcommand{\select}[1]{\fcolorbox{black}{CodeLightBlue}{#1}}
% A version of select without any space around the box.
% Hard to see in black & white so I chose the one with the space
\newcommand{\select}[1]{{\setlength{\fboxsep}{0.05cm}\fcolorbox{black}{CodeLightBlue}{#1}}}

\newcommand{\Red}[1]{{\color{red} #1}}
\newcommand{\TODO}[1]{{\color{red}{[TODO: #1]}}}
\newcommand{\FYI}[1]{{\color{green}{[FYI: #1]}}}
\newcommand{\forceindent}{\hspace{\parindent}}

\usepackage[normalem]{ulem}
\makeatletter
\newcommand\colorwave[1][blue]{\bgroup \markoverwith{\lower3.5\p@\hbox{\sixly \textcolor{#1}{\char58}}}\ULon}
\font\sixly=lasy6 % does not re-load if already loaded, so no memory problem.
\makeatother
\newcommand{\CodeError}[1]{\colorwave[red]{#1}}

\newcommand\mlnode[1]{
  \begin{minipage}{1.5cm}
    \linespread{1}\selectfont
    \vspace{0.3cm}
    \begin{center}
      #1
    \end{center}
    \vspace{0.3cm}
  \end{minipage}}

\sloppy
\hyphenpenalty=10000

%% Get rid of sloppy line breaks in \citet citations
\bibpunct{\nolinebreak[4][}{]}{,}{n}{}{,}
\makeatletter
\renewcommand*{\NAT@spacechar}{~}
\makeatother


\setenumerate{label=(\arabic*), parsep=0pt}

\newcommand\blfootnote[1]{%
  \begingroup
  \renewcommand\thefootnote{}\footnote{#1}%
  \addtocounter{footnote}{-1}%
  \endgroup
}

\begin{document}
\title{Semiotics of Metaprogramming}
\subtitle{}

\author{Joomy Korkut}
\orcid{0000-0001-6784-7108}
\affiliation{
  \institution{Princeton University}
  \city{Princeton}
  \state{New Jersey}
  \country{USA}
}
\email{joomy@cs.princeton.edu}

\renewcommand{\shortauthors}{Joomy Korkut}

\newcommand{\lc}{\mbox{\lam-calculus}}
\newcommand{\Lc}{\mbox{\Lam-calculus}}

% \begin{abstract}
% Haskell is a popular language for language implementations. However, adding metaprogramming features to a language implemented in Haskell requires a significant amount of boilerplate code. Using Data and Typeable style of generic programming in Haskell, we describe a design pattern that allows automatic derivation of metaprogramming features from your language implementation. If you have evaluation, you can evaluate quasiquoted terms for free. If you have type-checking, you can type-check quasiquoted terms for free. If you have a parser, you can have parser reflection for free. This design pattern is applicable to both untyped and typed languages, and can implement various features of metaprogramming.
% \end{abstract}

% The code below should be generated by the tool at
% http://dl.acm.org/ccs.cfm
% Please copy and insert the code instead of the example below.
% \begin{CCSXML}
% <ccs2012>
% <concept>
% <concept_id>10011007.10011006.10011008.10011009.10011012</concept_id>
% <concept_desc>Software and its engineering~Functional languages</concept_desc>
% <concept_significance>300</concept_significance>
% </concept>
% <concept>
% <concept_id>10011007.10011006.10011008.10011024.10011028</concept_id>
% <concept_desc>Software and its engineering~Data types and structures</concept_desc>
% <concept_significance>300</concept_significance>
% </concept>
% </ccs2012>
% \end{CCSXML}

% \ccsdesc[300]{Software and its engineering~Functional languages}
% \ccsdesc[300]{Software and its engineering~Data types and structures}

% \keywords{Metaprogramming, generic programming, reflection, Haskell.}

\maketitle
\thispagestyle{empty}


% \emph{This paper uses colors in the example code.}

\section{Introduction}



Haskell is considered to be one of the best in class for language implementations~\cite{sotu}.

% \section{Semiotic Foundations of Metaprogramming}

In the semiotics theory of Charles Sanders Peirce, a sign consists of three elements:

\begin{tikzcd}
\mlnode{Representamen}
  \arrow[rrrr, sloped, above, "\text{materializes into}" description, shift left=3]
  \arrow[rrdddd, sloped, above, "\text{decodes into}" description, shift left=3] &  &  &  &
\mlnode{Object}
  \arrow[llll, sloped, below, "\text{is represented by}" description, shift left=3]
  \arrow[lldddd, sloped, below, "\text{indicates}" description, shift left=3] \\
&  &  &  & \\
&  &  &  & \\
&  &  &  & \\
&  &
\mlnode{Interpretant}
  \arrow[lluuuu, sloped, below, "\text{encodes into}" description, shift left=3]
  \arrow[rruuuu, sloped, above, "\text{is denoted by}" description, shift left=3]
&  &
\end{tikzcd}

An interpretant is a sign's effect on the mind. For example, a stop sign in the traffic would be the representamen, and its object would the stop rule. Its interpretant would be the thought that you should stop there.

The intuition behind applying semiotic theory to metaprogramming is that your interpreter and compiler can also be considered a mind.

\begin{tikzcd}
\mlnode{object language term}
  \arrow[rrrr, sloped, above, "\text{materializes into}" description, shift left=3]
  \arrow[rrdddd, sloped, above, "\text{decodes into}" description, shift left=3] &  &  &  &
\mlnode{represented entity}
  \arrow[llll, sloped, below, "\text{is represented by}" description, shift left=3]
  \arrow[lldddd, sloped, below, "\text{indicates}" description, shift left=3] \\
&  &  &  & \\
&  &  &  & \\
&  &  &  & \\
&  &
\mlnode{host language term}
  \arrow[lluuuu, sloped, below, "\text{encodes into}" description, shift left=3]
  \arrow[rruuuu, sloped, above, "\text{is denoted by}" description, shift left=3]
&  &
\end{tikzcd}




% \section*{Acknowledgments}
% We would like to thank Matt Chan and Gabriel Gonzalez for inspiring the idea of
% a bridge between the meta language and the object language.
% We would also like to thank Charlie Murphy for his help with the draft.
% \vspace{0.5em}

\bibliographystyle{ACM-Reference-Format}
\bibliography{paper}

\end{document}
