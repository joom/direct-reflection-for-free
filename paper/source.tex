\documentclass[sigplan, authordraft]{acmart}

% Copyright
%\setcopyright{none}
%\setcopyright{acmcopyright}
\setcopyright{acmlicensed}
%\setcopyright{rightsretained}
%\setcopyright{usgov}
%\setcopyright{usgovmixed}
%\setcopyright{cagov}
%\setcopyright{cagovmixed}

% DOI
\acmDOI{}

% ISBN
\acmISBN{}

%Conference
\acmConference[ICFP'19]{the 24th ACM SIGPLAN International Conference on Functional Programming}{August 18-25, 2019}{Berlin, Germany}
\acmYear{2019}
\copyrightyear{2019}
\acmPrice{15.00}
\acmSubmissionID{}

\usepackage{enumerate, enumitem}
\usepackage{fancyvrb}
\fvset{commandchars=\\\{\}}
\usepackage{xcolor}
\usepackage{setspace}
\usepackage{todonotes}
\usepackage{tikz-cd}
\usepackage{adjustbox}
% Idris
% in REPL run `:pp latex 80 <expr>` to get the colored verbatim

\usepackage{amssymb}
\newcommand\fakeslant[1]{%
  \special{pdf: literal 1 0 0.167 1 0 0 cm}#1\special{pdf: literal 1 0 -0.167 1 0 0 cm}}
\definecolor{IdrisRed}{HTML}{BC0045}
\definecolor{IdrisBlue}{HTML}{0000d6}
\definecolor{IdrisGreen}{HTML}{008b00}
\definecolor{IdrisLilac}{HTML}{AC22BF}
\definecolor{IdrisYellow}{HTML}{E16800}
\definecolor{IdrisLightBlue}{HTML}{A4CDFF}
\newcommand{\IdrisData}[1]{\textcolor{IdrisRed}{#1}}
\newcommand{\IdrisType}[1]{\textcolor{IdrisBlue}{#1}}
\newcommand{\IdrisBound}[1]{\textcolor{IdrisLilac}{\fakeslant{#1}}}
\newcommand{\IdrisFunction}[1]{\textcolor{IdrisGreen}{#1}}
\newcommand{\IdrisMetavar}[1]{\textcolor{IdrisYellow}{#1}}
\newcommand{\IdrisKeyword}[1]{{\textbf{#1}}}
\newcommand{\IdrisImplicit}[1]{{\itshape \IdrisBound{#1}}}
\newcommand{\ty}[1]{\IdrisType{\texttt{#1}}}
\newcommand{\kw}[1]{\IdrisKeyword{\texttt{#1}}}
\newcommand{\fn}[1]{\IdrisFunction{\texttt{#1}}}
\newcommand{\dt}[1]{\IdrisData{\texttt{#1}}}
\newcommand{\bn}[1]{\IdrisBound{\texttt{#1}}}
\newcommand{\cm}[1]{\textcolor{darkgray}{\texttt{#1}}}
\newcommand{\hole}[1]{\IdrisMetavar{\texttt{?}\IdrisMetavar{\texttt{#1}}}}
\newcommand{\Elab}{\ty{Elab}}
\newcommand{\IO}{\ty{IO}}
\newcommand{\String}{\ty{String}}
\newcommand{\TT}{\ty{TT}}
\newcommand{\Raw}{\ty{Raw}}
\newcommand{\Editorable}{\ty{Editorable}}
\newcommand{\TyDecl}{\ty{TyDecl}}
\newcommand{\FunDefn}{\ty{FunDefn}}
\newcommand{\FunClause}{\ty{FunClause}}
\newcommand{\Edit}{\ty{Edit}}
\newcommand{\sexp}{\mbox{S-expression}} %don't hyphenate
\newcommand{\mt}[1]{\mbox{\texttt{#1}}}
\newcommand{\Prim}[1]{\dt{Prim\_\_#1}}
\newcommand{\qt}[1]{\kw{\`{}(}{#1}\kw{)}} %quotation `(...)
\newcommand{\antiqt}{\kw{\textasciitilde}} %antiquotation ~

% Replicating selecting regions in the editor:
% \newcommand{\select}[1]{\fcolorbox{black}{IdrisLightBlue}{#1}}
% A version of select without any space around the box.
% Hard to see in black & white so I chose the one with the space
\newcommand{\select}[1]{{\setlength{\fboxsep}{0.05cm}\fcolorbox{black}{IdrisLightBlue}{#1}}}

\newcommand{\Red}[1]{{\color{red} #1}}
\newcommand{\TODO}[1]{{\color{red}{[TODO: #1]}}}
\newcommand{\FYI}[1]{{\color{green}{[FYI: #1]}}}
\newcommand{\forceindent}{\hspace{\parindent}}

\usepackage[normalem]{ulem}
\makeatletter
\newcommand\colorwave[1][blue]{\bgroup \markoverwith{\lower3.5\p@\hbox{\sixly \textcolor{#1}{\char58}}}\ULon}
\font\sixly=lasy6 % does not re-load if already loaded, so no memory problem.
\makeatother
\newcommand{\IdrisError}[1]{\colorwave[red]{#1}}

\sloppy

%% Get rid of sloppy line breaks in \citet citations
\bibpunct{\nolinebreak[4][}{]}{,}{n}{}{,}
\makeatletter
\renewcommand*{\NAT@spacechar}{~}
\makeatother


\setenumerate{label=(\arabic*), parsep=0pt}

\begin{document}
\title{Direct Reflection for Free!}
\subtitle{}

\author{Joomy Korkut}
\orcid{0000-0001-6784-7108}
\affiliation{
  \institution{Princeton University}
  \city{Princeton}
  \state{New Jersey}
  \country{USA}
}
\email{joomy@cs.princeton.edu}

\renewcommand{\shortauthors}{Joomy Korkut}

\begin{abstract}
  For almost a decade now, Haskell has been equipped for the ability to derive the \ty{Generic} type class, which describes the shape of a data type~\cite{magalhaes2010generic}.

\end{abstract}

%
% The code below should be generated by the tool at
% http://dl.acm.org/ccs.cfm
% Please copy and insert the code instead of the example below.
%
 \begin{CCSXML}
<ccs2012>
<concept>
<concept_id>10011007.10011006.10011008.10011009.10011012</concept_id>
<concept_desc>Software and its engineering~Functional languages</concept_desc>
<concept_significance>300</concept_significance>
</concept>
<concept>
<concept_id>10011007.10011006.10011008.10011024.10011028</concept_id>
<concept_desc>Software and its engineering~Data types and structures</concept_desc>
<concept_significance>300</concept_significance>
</concept>
</ccs2012>
\end{CCSXML}

\ccsdesc[300]{Software and its engineering~Functional languages}
\ccsdesc[300]{Software and its engineering~Data types and structures}

\keywords{Metaprogramming, generic programming, reflection}

\maketitle

\emph{This paper uses colors in the example code.}

\section{Introduction}


\TODO{how we can map Generic to data types in the object language}

\section{Problem}

Implementing metaprogramming, even for toy languages, has been a cumbersome task that makes maintenance much more costly.
If we are working with a typed meta language and another typed object language, then one often needs to redefine the AST data type in the object language, so that we would have a syntax tree representation in the object language.
The problem with this is that one has to keep the AST in the compiler implementation and the AST representative data type in the object language, which we can call the ``reflected AST'', in sync.
Even though defining these separately has advantages, it also leads to high code duplication.
Ideally, we would like to have a two-way mapping between the meta language and the object language, which would save us from the code duplication.

% \input{introduction}
% \input{design}
% \input{implementation}
% \input{applications}
% \input{relatedwork}
% \input{conclusion}

\section*{Acknowledgments}

We would like to thank Matt Chan and Gabriel Gonzalez for inspiring the idea of a bridge between the meta language and the object language.

\bibliographystyle{ACM-Reference-Format}
\bibliography{paper}

\end{document}
